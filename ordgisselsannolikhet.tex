\documentclass{article}
\usepackage[utf8]{inputenc}
\usepackage{amsmath, amssymb}
\begin{document}
\paragraph{Premisser.}
Antal ord att gissa på är $N$.
Strategin är att ta mittersta ordet i möjliga listan.
Med 
$n = \log(N+1)/\log(2)$.
är
$N=2^n-1$.
Vi förutsätter i beräkningarna att $n$ är ett heltal, så att strategin kan följas entydigt i rekursion, då $2^n-1=2(2^{n-1}-1)+1$.

\paragraph{Slutsatser.}
I rekursionen minskas $n$ med ett i varje steg.
Med premisserna ovan behövs det i sämsta fall $n$ gissningar, eftersom $2^{1}-1=1$.
Sannolikheten för att behöva exakt $n$ gissningar är approximativt en på två. Mer specifikt, för sannolikheten $p_k$ att gissa rätt på försök $k$ har vi
\begin{align*}
p_1 &= \frac{1}{N} = \frac{1}{2^n-1} \\
p_2 &= (1-p_1)\, \frac{1}{(2^n-2)/2} = \frac{2^n-2}{2^n-1}\frac{2}{2^n-2} = 2\,p_1 \\
&\vdots \\
p_k &= 2\,p_{k-1} = 2^{k-1}p_1 \\
&\vdots \\
p_n &= 2^{n-1}p_1 = \frac{1}{2-2^(-n+1)} \approx \frac{1}{2}.
\end{align*}
Väntevärdet på antalet gissningar är
\[
\sum_{k=1}^n n\,p_n 
=
p_1\sum_{k=1}^n k 2^{k-1} = p_1\left((n-1)2^n+1\right)
= 
\frac{(n-1)2^n+1}{2^n-1}
=
\frac{(n-1)2^{-n}}{1-2^{-n}}\approx n-1.
\]




\paragraph{Appendix.}
\[\begin{split}
S_n&=\sum_{k=1}^{n}k 2^{k-1}\\
\implies
2S_n &= \sum_{k=1}^{n}k 2^{k}
= \sum_{k=0}^{n}k 2^{k}
= \sum_{i=1}^{n+1} (i-1) 2^{i-1}
= \sum_{i=1}^{n+1} i 2^{i-1} 
- \sum_{i=1}^{n+1} 2^{i-1}
\\&= S_n + (n+1)2^{n} - 2^{n+1}
= S_n + (n-1)2^{n-1}
\end{split}\]
\[\therefore \sum_{k=1}^{n}k 2^{k-1} = (n-1)2^{n-1}.\]


\end{document}
